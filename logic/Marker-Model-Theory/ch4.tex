\documentclass[10pt]{article}
\usepackage{graphicx}    % needed for including graphics e.g. EPS, PS
\topmargin -1.5cm        % read Lamport p.163
\oddsidemargin -0.04cm   % read Lamport p.163
\evensidemargin -0.04cm  % same as oddsidemargin but for left-hand pages
\textwidth 16.59cm
\textheight 21.94cm 
\parskip 0pt            % sets spacing between paragraphs
\parindent 0pt		     % sets leading space for paragraphs
\usepackage{mathtools}
\DeclarePairedDelimiterX\Set[2]{\lbrace}{\rbrace}
 { #1 \,\delimsize|\, #2 }
\usepackage{amsthm}
\usepackage{amssymb}
\usepackage{amsmath}
\usepackage{enumerate}
\newcommand{\after}{\circ } 
\newcommand{\la}{\langle } 
\newcommand{\ra}{\rangle } 
\newcommand{\ti}{\to \infty} 
\newcommand{\R}{\mathbb{R}}
\newcommand{\C}{\mathbb{C}}
\newcommand{\Z}{\mathbb{Z}}
\newcommand{\N}{\mathbb{N}}
\newcommand{\Q}{\mathbb{Q}}
\newcommand{\F}{\mathbb{F}}
\newcommand{\e}{\varepsilon}
\newcommand{\A}{\forall}
\newcommand{\mcC}{\mathcal{C}} 
\newcommand{\satisfies}{\models} 
\newcommand{\Mod}{\text{Mod}} 
\newcommand{\inn}{\varepsilon}
\newcommand{\ninn}{\not\varepsilon}
\newcommand{\E}{\exists}
\newcommand{\mfC}{\mathfrak{C}}
\newcommand{\mfG}{\mathfrak{G}}
\newcommand{\mfN}{\mathfrak{N}}
\newcommand{\mfR}{\mathfrak{R}}
\newcommand{\mcE}{\mathcal{E}}
\newcommand{\Hom}{\text{Hom}}
\newcommand{\Id}{\text{Id}}
\newcommand{\inv}{^{-1}}
\newcommand{\s}{\sqrt}
\newcommand{\half}{\frac{1}{2}}

\usepackage{tikz}
\usetikzlibrary{matrix,arrows,automata}
\newtheorem{lem}{Lemma}
\author{Adam Freilich}
\title{Marker's Model Theory Problems \S 4.4}
\begin{document}
\maketitle

\begin{enumerate}[1.]
%1
\item
 
  \begin{enumerate}[a)] 
  \item 	
  \end{enumerate}

\item
 
  \begin{enumerate}[a)] 
  \item 	
  \end{enumerate}

\item
 
  \begin{enumerate}[a)] 
  \item 	
  \end{enumerate}

\item
 
  \begin{enumerate}[a)] 
  \item 	
  \end{enumerate}

\item
 
  \begin{enumerate}[a)] 
  \item 	
  \end{enumerate}

%6
\item
 
  \begin{enumerate}[a)] 
  \item 	
  \end{enumerate}

\item
 
  \begin{enumerate}[a)] 
  \item 	
  \end{enumerate}

\item
 
  \begin{enumerate}[a)] 
  \item 	
  \end{enumerate}

\item
 
  \begin{enumerate}[a)] 
  \item 	
  \end{enumerate}

\item
 
  \begin{enumerate}[a)] 
  \item 	
  \end{enumerate}

%11
\item
 
  \begin{enumerate}[a)] 
  \item 	
  \end{enumerate}

\item
 
  \begin{enumerate}[a)] 
  \item 	
  \end{enumerate}

\item
 
  \begin{enumerate}[a)] 
  \item 	
  \end{enumerate}

\item
 
  \begin{enumerate}[a)] 
  \item 	
  \end{enumerate}

\item
 
  \begin{enumerate}[a)] 
  \item 	
  \end{enumerate}

%16
\item
 
  \begin{enumerate}[a)] 
  \item 	
  \end{enumerate}

\item
 
  \begin{enumerate}[a)] 
  \item 	
  \end{enumerate}

\item
 
  \begin{enumerate}[a)] 
  \item 	
  \end{enumerate}

\item
 
  \begin{enumerate}[a)] 
  \item 	
  \end{enumerate}

\item
 
  \begin{enumerate}[a)] 
  \item 	
  \end{enumerate}

%21
\item
 
  \begin{enumerate}[a)] 
  \item 	
  \end{enumerate}

\item
 
  \begin{enumerate}[a)] 
  \item 	
  \end{enumerate}

\item
 
  \begin{enumerate}[a)] 
  \item 	
  \end{enumerate}

\item
 
  \begin{enumerate}[a)] 
  \item 	
  \end{enumerate}

\item
 
  \begin{enumerate}[a)] 
  \item 	
  \end{enumerate}

%26
\item
 
  \begin{enumerate}[a)] 
  \item 	
  \end{enumerate}

\item
 
  \begin{enumerate}[a)] 
  \item 	
  \end{enumerate}

\item
 
  \begin{enumerate}[a)] 
  \item 	
  \end{enumerate}

\item
 
  \begin{enumerate}[a)] 
  \item 	
  \end{enumerate}

\item
 
  \begin{enumerate}[a)] 
  \item 	
  \end{enumerate}

%31
\item
 
  \begin{enumerate}[a)] 
  \item 	
  \end{enumerate}

\item
 
  \begin{enumerate}[a)] 
  \item 	
  \end{enumerate}

\item
 
  \begin{enumerate}[a)] 
  \item 	
  \end{enumerate}

\item
 
  \begin{enumerate}[a)] 
  \item 	
  \end{enumerate}

\item
 
  \begin{enumerate}[a)] 
  \item 	
  \end{enumerate}

%36
\item
 
  \begin{enumerate}[a)] 
  \item 	
  \end{enumerate}

\end{enumerate}

\end{document}
