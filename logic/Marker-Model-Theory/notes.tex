\documentclass[10pt]{article}
\usepackage{graphicx}    % needed for including graphics e.g. EPS, PS
\topmargin -1.5cm        % read Lamport p.163
\oddsidemargin -0.04cm   % read Lamport p.163
\evensidemargin -0.04cm  % same as oddsidemargin but for left-hand pages
\textwidth 16.59cm
\textheight 21.94cm 
\parskip 0pt            % sets spacing between paragraphs
\parindent 0pt		     % sets leading space for paragraphs
\usepackage{mathtools}
\DeclarePairedDelimiterX\Set[2]{\lbrace}{\rbrace}
 { #1 \,\delimsize|\, #2 }
\usepackage{amsthm}
\usepackage{amssymb}
\usepackage{amsmath}
\usepackage{enumerate}
\newcommand{\after}{\circ } 
\newcommand{\la}{\langle } 
\newcommand{\ra}{\rangle } 
\newcommand{\ti}{\to \infty} 
\newcommand{\R}{\mathbb{R}}
\newcommand{\C}{\mathbb{C}}
\newcommand{\Z}{\mathbb{Z}}
\newcommand{\N}{\mathbb{N}}
\newcommand{\Q}{\mathbb{Q}}
\newcommand{\F}{\mathbb{F}}
\newcommand{\e}{\varepsilon}
\newcommand{\A}{\forall}
\newcommand{\mcC}{\mathcal{C}} 
\newcommand{\satisfies}{\models} 
\newcommand{\Mod}{\text{Mod}} 
\newcommand{\inn}{\varepsilon}
\newcommand{\ninn}{\not\varepsilon}
\newcommand{\E}{\exists}
\newcommand{\mfC}{\mathfrak{C}}
\newcommand{\mfG}{\mathfrak{G}}
\newcommand{\mfN}{\mathfrak{N}}
\newcommand{\mfR}{\mathfrak{R}}
\newcommand{\mcE}{\mathcal{E}}
\newcommand{\mcA}{\mathcal{A}}
\newcommand{\mcL}{\mathcal{L}}
\newcommand{\mcM}{\mathcal{M}}
\newcommand{\mcN}{\mathcal{N}}
\newcommand{\qf}{quantifier-free }
\newcommand{\qe}{quantifier elimination }
\newcommand{\Hom}{\text{Hom}}
\newcommand{\Id}{\text{Id}}
\newcommand{\inv}{^{-1}}
\newcommand{\s}{\sqrt}
\newcommand{\half}{\frac{1}{2}}

\usepackage{tikz}
\usetikzlibrary{matrix,arrows,automata}
\newtheorem{lem}{Lemma}
\author{Adam Freilich}
\title{Marker's Model Theory Notes}
\begin{document}
\maketitle

\section{Chapter 1}
\section{Chapter 2}
\section{Chapter 3}
Ways to show \(T\) admits quantifier elimination (i.e. sufficient conditions):

\begin{itemize}
\item Directly (but hopefully not)
\item Show that \textbf{if} \(\phi\) is \qf and \(\mcM, \mcN \models T \text{ and } \mcM, \mcN \supset \mcA \text{ and } \bar{a} \in \mcA^n\) 

then \(\mcM \models \exists x \phi(\bar{a}, x) \iff \mcN \models \exists x \phi(\bar{a}, x\))

(The proof relies on two facts: to show \qe it suffices to show for existentials of depth 1 and that a formula being \qf is equivalent to this substructure definition)

\item Show that \(T\) has algebraically prime models and that if \(\mcM, \mcN \models T, \mcM \subset \mcN\) that \(\mcM \prec_s \mcN\) (all substructures are simply closed in their extensions).

Where \(T\) has algebraically prime models when any model \(\mcM \models T_\A\) can be embedded in a \(\mcN \models T\) which embeds in any other such extension of \(\mcM\). 

And where \(\mcM \prec_s \mcN\) when, for any \(\phi = \E x \psi(\bar{a}, x)\) for \(\psi\) \qf, then \(\mcM \models \phi \iff \mcN \models \phi\) (the extension makes no more quantifier depth 1 existential formulae true).

\end{itemize}

Now for some necessary conditions: 
\begin{itemize}

\item I beleive the second condition above is equivalent (more or less that existentials admit \qe)

\item Model completeness: All embeddings are elementary. 

\end{itemize}


\end{document}
