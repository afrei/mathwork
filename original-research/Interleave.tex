\documentclass[10pt]{IEEEtran}%IEEEtran
\usepackage{graphicx}    % needed for including graphics e.g. EPS, PS
\topmargin -1.5cm        % read Lamport p.163
\oddsidemargin -0.04cm   % read Lamport p.163
\evensidemargin -0.04cm  % same as oddsidemargin but for left-hand pages
\textwidth 16.59cm
\textheight 21.94cm 
                         %\pagestyle{empty}       
						 % Uncomment if don't want page numbers
\parskip 5pt            % sets spacing between paragraphs
                         %\renewcommand{\baselinestretch}{1.5} 
						 % Uncomment for 1.5 spacing between lines
\parindent 5pt		     % sets leading space for paragraphs
\usepackage{mathtools}
\DeclarePairedDelimiterX\Set[2]{\lbrace}{\rbrace}
 { #1 \,\delimsize|\, #2 }
\usepackage{amsthm}
\usepackage{amssymb}
\usepackage{amsmath}
\usepackage{hyperref}
\usepackage{enumerate}
\newcommand{\after}{\circ } 
\newcommand{\la}{\langle } 
\newcommand{\ra}{\rangle } 
\newcommand{\ti}{\to \infty} 
\newcommand{\R}{\mathbb{R}}
\newcommand{\C}{\mathbb{C}}
\newcommand{\Z}{\mathbb{Z}}
\newcommand{\N}{\mathbb{N}}
\newcommand{\Q}{\mathbb{Q}}
\newcommand{\F}{\mathbb{F}}
\newcommand{\e}{\varepsilon}
\newcommand{\A}{\forall}
\newcommand{\mcC}{\mathcal{C}} 
\newcommand{\rmn}{Riemann integral }
\newcommand{\lbg}{Lesbegue }
\newcommand{\satisfies}{\models} 
\newcommand{\Mod}{\text{Mod}} 
\newcommand{\inn}{\varepsilon}
\newcommand{\ninn}{\not\varepsilon}
\newcommand{\E}{\exists}
\newcommand{\mfC}{\mathfrak{C}}
\newcommand{\mfG}{\mathfrak{G}}
\newcommand{\mfN}{\mathfrak{N}}
\newcommand{\mfR}{\mathfrak{R}}
\newcommand{\mcE}{\mathcal{E}}
\newcommand{\p}{\partial}
\newcommand{\tens}{\otimes}
\newcommand{\Hom}{\text{Hom}}
\newcommand{\Id}{\text{Id}}
\newcommand{\inv}{^{-1}}
\newcommand{\s}{\sqrt}
\newcommand{\half}{\frac{1}{2}}

\usepackage{tikz}
\usetikzlibrary{matrix,arrows,automata}
\newtheorem{lem}{Lemma}
\author{Adam Freilich}
\title{Interleave isn't regular}
\begin{document}

%\maketitle

Consider the partial function we will call ``interleave'' which is defined on a subset of $\{ a, b, \# \}^*$. More specifically, it is defined on strings which can be written as $u_1\ldots u_n\#v_1 \ldots v_n$ where all $u_i, v_i\in \{a, b\}$. And \[\text{inteleave}(u_1\ldots u_n\#v_1 \ldots v_n) = u_1v_1u_2v_2\ldots u_nv_n\]

Less formally, the function takes two strings seperated by a hash and creates a string by alternating letters from the first and second. We aim to show that no function agreeing with interleave where interleave is defined is regular. 

We prove something a little stronger, which is that no function agreeing with interleave on strings of $a^*b^*\#a^*b^*$ where interleave is defined is regular. 

We will use the MSO-dfinability critereon for regular functions to show this. We recall this definition (as given by Engelfreit and Hoogeboom) at the very end.


Our intuition should serve us well here. ``local formulae'' within the first of the two strings given as input to interleave, tell us very little about the output. Almost everywhere in the output, we depend on information from both the first and second string. Similarly, ``long distance formulae'', which extend from one side of the hash to the other, can tell use very little about the output since, they must isolate a specific node from the first side to be connected to a given node on the other side. All of this will be made more precise now. 

We will associate edges in $\tau(G)$ with nodes in $G$ for such a transduction as follows: given two subsets, $S, T$ of $V_g$, and edge of $\tau(G)$ is an $(S, T)$-edge if it is of the form $(u^{c_1}, \gamma, v^{c_2})$ with $u \in S, v \in T$ or $u \in T, v \in S$. We will refer to $(S, S)$-edges as $S$-edges. 

Let $k$ be the highest rank of any of the formulae specifying an MSO tranformation $\tau$. Let $n$ be the number of rank $k$ types over the language of graphs described above\footnote{We may need a few constants or unary predicates. Ultimately, $n$ might be a little bigger than I just said, but it should be finite and work anyway, regardless of the exact value of $n$.}. Note this is finite (see proposition 4 at \url{http://www.cis.upenn.edu/~alur/Lics13inf.pdf}). Let $c$ be $|C|$ with $C$ the copy set mentioned above. 

Consider strings of the form $w = a^{n_1}b^{n_2}\#a^{n_3}b^{n_4}$ where $n_i$s are all at least $n$ and $n_1 + n_2 = n_3 + n_4$. We divide letters in this string into 3 sets. We can write $w = a^{n}w_1b^{n}\#a^{n}w_2b^{n}$, and our three sets will be $W_1$ which is the letters in $w_1$, $W_2$, those in $w_2$ and the rest, $R$, (of which there are always $4n+1$). We also define $W = W_1 \cup W_2 \cup R$, i.e. all of the letters in $w$. Note that all edges in the output are either: $(W, R)$-edges, $(W_1, W_2)$-edges [cross edges], or $W_1$-edges or $W_2$-edges [cis edges]. We will achieve a bound on the number of edges from an `a' node to a `b' node in each category which will not depend on the size of any of the $n_i$s. As the number of such edges in an output of interleave is unbounded, this will give us a contradiction. 

\textbf{Lemma 1:} Any graph in the image of $\tau$ has at most $2(4n +1)c$ $(W, R)$-edges. 

This follows from noting that every such edge has at least one endpoint which is a node in $R \times C$. There are $c|R|$ such nodes, and, should the output be a string, each can give rise to at most 2 edges. This gives us a bound of $2(4n +1)c$ such edges. 

\textbf{Lemma 2:} Any graph in the image of $\tau$ has at most $cn^2$ cross edges. 

For each letter $x$ in $w_1$ consider the rank $k$ type of $a^{n}w_1b^{n}$ where the added constant $p$ takes the value $x$. Do the same for each $y$ in $w_2$ (call this constant $q$). If any of these types is realized twice, then all such $(x, c)$ nodes cannot have any cross edges. If one such edge exists, $\phi^{c_1, c_2}(p, q)$ holds for each $x$ achieving that type, of which there are more than 1. This means that some node will have in degree or out degree at least 2, which cannot happen in string to string transformations. This gives us the $cn^2$ bound for cross edges.  

\textbf{Lemma 3:} For any graph $\tau(a^{n_1}b^{n_2}\#a^{n_3}b^{n_4})$ where $n_i$s are all at least $n$ and $n_1 + n_2 = n_3 + n_4$ there are at most $n^2$ cis edges $(x, \gamma, y)$ where $x$ is an `a' and $y$ is a `b'.

Here we use that $\cong_k$ is a monoid congruence (see proposition 4 at \url{http://www.cis.upenn.edu/~alur/Lics13inf.pdf}). Note that for some $m \leq n$ that $a^{n-m}\cong_k a^{n+im}$ for all $i \in \N$. and for some $m' \leq n$ that $a^{k-m}\cong_k a^{k+im}$ for all $i \in \N$. Assume that an `a' to `b' edge is present in $\tau(a^{n}w_1b^{n}\#a^{n}w_2b^{n})$, say it is $(x, \gamma, y)$. We add the constants $x, y$ to our language. $\phi_\gamma^{c_1, c_2}(x, y)$ holds and $\phi_a^{c_1}(x)$ and $\phi_b^{c_2}(y)$ do as well. They will appear in every $\tau(a^{n + imm'}w_1b^{n+jmm'}\#a^{n+jmm'}w_2b^{n + imm'})$ because this is $\cong_k$ to our original $a^{n}w_1b^{n}\#a^{n}w_2b^{n}$ with added constants $x, y$. Choosing $i, j$ properly, we can get the number of `a's in the first half and in the second half to be within $mm'$ of each other. In the resulting string, if $\tau$ is to agree with interleave, there must be at most $mm' < n^2$ `a' to `b' edges in the modified string (and therefore the original string as well). 

Taken together, we see that for any $\tau$ which agrees with interleave, it will have at most $2(4n+1)c + cn^2 + n^2$ edges from an `a' to a `b' on inputs of the form $a^{n}w_1b^{n}\#a^{n}w_2b^{n}$ where $w_1, w_2 \in a^*b^*$. But for sufficiently large strings of this form, like when $w_1 = a^{2cn^2}, w_2 = b^{2cn^2}$, there are more than $2(4n+1)c + cn^2 + n^2$ edges from an `a' to a `b'. This gives us our cantradiction. 

Note that in general, for any MSO tranformation $\tau$, lemmas 1 and 2 hold (where $R$ is finite and $W_1, W_2$ are disjoint). Lemma 3 holds here because, in some sense, local information is useless in determining the output of interleave. A function like the identity, which can have output of unbounded length, is still regular because we cannot achieve a bound on the number of useful cis edges. 

\newpage

An MSO definable graph transduction $\tau: GR(\Sigma_1, \Gamma_1) \to GR(\Sigma_2, \Gamma_2)$ ($GR(\Sigma, \Gamma)$ is graphs where nodes have labels in $\Sigma$ and edges have labels in $\Gamma$) is specified by 

a closed domain formula $\phi_{dom}$, 

a finite copy set $C$, 

node formulas $\phi_\sigma^c(x)$ with one free node variable $x$ for every $\sigma \in \Sigma_2$ and every $c \in C$, and

edge formulas $\phi_\gamma^{c_1, c_2}(x, y)$ with two free node variable $x, y$ for every $\gamma \in \Gamma_2$ and all $c_1, c_2 \in C$. 

where all formulas are in $MSO(\Sigma_1, \Gamma_1)$. 

For $g \in G(\phi_{dom})$ with node set $V_g$, the image $\tau(g)$ is the graph $(V, E, l)$ defined as follows. We will write $u^c$ rather than $(u, c)$ for elements of $V_g \times C$. 

$V = \{u^c | u \in V_g, c \in C, \text{ there is exactly one } \sigma \in \Sigma_2 \text{ such that } g\models \phi_\sigma^c(u)\}$,

$E = \{(u^{c_1}, \gamma, v^{c_2}) | u^{c_1}, v^{c_2} \in V, \gamma \in \Gamma_2, \\ g \models \phi_\gamma^{c_1, c_2}(u, v)\}$ and

$l(u^c) = \sigma$ if $g \models \phi_\sigma^c(u)$ for $u^c \in V, \sigma \in \Sigma_2$.

 

\end{document}